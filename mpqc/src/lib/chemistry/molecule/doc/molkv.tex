
%%%%%%%%%%%%%%%%%%%%%%%%%%%%%%%%%%%%%%%%%%%%%%%%%%%%%%%%%%%%%%%%%%%%%%%%%%

\section{The \clsnm{Molecule} Class}\label{Molecule}\index{Molecule}

The \clsnm{Molecule} class contains information about molecules.  It has a
\clsnmref{KeyVal} constructor that can create a new molecule from either a
PDB file or from a list of Cartesian coordinates.

The following \clsnmref{ParsedKeyVal} input reads from the PDB
file \verb|h2o.pdb|:
\begin{alltt}
molecule<\clsnmref{Molecule}>: (
   pdb_file = "h2o.pdb"
 )
\end{alltt}

The following input explicitly gives the atom coordinates, using the
\clsnmref{ParsedKeyVal} \htmlref{table}{pkvtable} notation:
\begin{alltt}
molecule<\clsnmref{Molecule}>: (
    angstrom=yes
    \{ atom_labels atoms           geometry            \} = \{
          O1         O   [ 0.000000000 0  0.369372944 ]
          H1         H   [ 0.783975899 0 -0.184686472 ]
          H2         H   [-0.783975899 0 -0.184686472 ]
     \}
    )
  )
\end{alltt}
The default units are Bohr with can be overridden with
\verb|angstrom=yes|.  The \verb|atom_labels| array can be
omitted.  The \verb|atoms| and \verb|geometry| arrays
are required.

The \clsnmref{Molecule} class has a \clsnmref{PointGroup}
member object, which also has a \clsnmref{KeyVal} constructor
that is called when a \clsnmref{Molecule} is made.  The
following example constructs a molecule with $C_{2v}$ symmetry:
\begin{alltt}
molecule<\clsnmref{Molecule}>: (
    symmetry=c2v
    angstrom=yes
    \{ atoms         geometry            \} = \{
        O   [0.000000000 0  0.369372944 ]
        H   [0.783975899 0 -0.184686472 ]
     \}
    )
  )
\end{alltt}
Only the symmetry unique atoms need can be specified.  Nonunique
atoms can be given too, however, numerical errors in the
geometry specification can result in the generation of extra
atoms so be careful.

%%%%%%%%%%%%%%%%%%%%%%%%%%%%%%%%%%%%%%%%%%%%%%%%%%%%%%%%%%%%%%%%%%%%%%%%%%

\section{The \clsnm{MolecularEnergy} Class}\label{MolecularEnergy}\index{MolecularEnergy}

The \clsnm{MolecularEnergy} abstract class inherits from the
\clsnmref{Function} class.  It computes the energy of the
molecule as a function of the geometry.  The coordinate system
used can be either internal or cartesian.

\begin{description}
  \item[\keywd{molecule}] A \clsnmref{Molecule}
    \htmlref{object}{pkvobject}.  There is no default.

  \item[\keywd{coor}] A \clsnmref{MolecularCoor}
    \htmlref{object}{pkvobject} that describes the coordinates.  If this is
    not given cartesian coordinates will be used.  For convenience, two
    keywords needed by the \clsnmref{MolecularCoor} object are
    automatically provided: \keywd{natom3} and \keywd{matrixkit}.

  \item[\keywd{value\_accuracy}] Sets the accuracy to which values are
    computed.  The default is 1.0e-6 atomic units.

  \item[\keywd{gradient\_accuracy}] Sets the accuracy to which gradients
    are computed.  The default is 1.0e-6 atomic units.

  \item[\keywd{hessian\_accuracy}] Sets the accuracy to which hessians are
    computed.  The default is 1.0e-4 atomic units.

  \item[\keywd{print\_molecule\_when\_changed}] If true, then whenever the
    molecule's coordinates are updated they will be printed.  The default
    is true.
\end{description}

%%%%%%%%%%%%%%%%%%%%%%%%%%%%%%%%%%%%%%%%%%%%%%%%%%%%%%%%%%%%%%%%%%%%%%%%%%

\section{The \clsnm{MolecularCoor} Class}\label{MolecularCoor}\index{MolecularCoor}

The \clsnm{MolecularCoor} abstract class describes the coordinate system
used to describe a molecule.  It is used to convert a molecule's cartesian
coordinates to and from this coordinate system.

\begin{description}
  \item[\keywd{molecule}] A \clsnmref{Molecule}
    \htmlref{object}{pkvobject}.  There is no default.

  \item[\keywd{debug}] An integer which, if nonzero, will cause extra
    output.

  \item[\keywd{matrixkit}] A \clsnmref{SCMatrixKit}
    \htmlref{object}{pkvobject}.  It is usually unnecessary to give this
    keyword.

  \item[\keywd{natom3}] An \clsnmref{SCDimension}
    \htmlref{object}{pkvobject} for the dimension of the vector of
    cartesian coordinates.  It is usually unnecessary to give this keyword.
\end{description}

%%%%%%%%%%%%%%%%%%%%%%%%%%%%%%%%%%%%%%%%%%%%%%%%%%%%%%%%%%%%%%%%%%%%%%%%%%

\section{The \clsnm{IntMolecularCoor} Class}\label{IntMolecularCoor}\index{IntMolecularCoor}

The \clsnm{IntMolecularCoor} abstract class inherits from the
\clsnmref{MolecularCoor} class.  It describes a molecule's coordinates in
terms of internal coordinates.

\begin{description}
  \item[\keywd{variable}] Gives a \clsnmref{SetIntCoor}
    \htmlref{object}{pkvobject} that specifies the internal coordinates
    that can be varied. If this is not given, the variable coordinates will
    be generated.

  \item[\keywd{followed}] Gives a \clsnmref{SetIntCoor}
    \htmlref{object}{pkvobject} that specifies coordinates to include in
    the variable coordinate list.  The remaining coordinates will be
    automatically generated.  The default is no followed coordinates.

  \item[\keywd{fixed}] Gives a \clsnmref{SetIntCoor}
    \htmlref{object}{pkvobject} that specifies the internal coordinates
    that will be fixed.  The default is no fixed coordinates.

  \item[\keywd{have\_fixed\_values}] If true, then values for the fixed
    coordinates must be given in \keywd{fixed} and an attempt will be made
    to displace the initial geometry to the given fixed values. The default
    is false.

  \item[\keywd{extra\_bonds}] This is only read if the \keywd{generator}
     keyword is not given.  It is a \htmlref{vector}{pkvarray} of atom
     numbers, where elements $(i-1)\times 2 + 1$ and $i\times 2$ specify
     the atoms which are bound in extra bond $i$.  The \keywd{extra\_bonds}
     keyword should only be needed for weakly interacting fragments,
     otherwise all the needed bonds will be found.

  \item[\keywd{generator}] Specifies an \clsnmref{IntCoorGen}
    \htmlref{object}{pkvobject} that creates simple, redundant internal
    coordinates. If this keyword is not given, then a vector giving extra
    bonds to be added is read from \keywd{extra\_bonds} and this is used to
    create an \clsnmref{IntCoorGen} object.

  \item[\keywd{decouple\_bonds}] Automatically generated internal
    coordinates are linear combinations of possibly any mix of simple
    internal coordinates.  If \keywd{decouple\_bonds} is true, an attempt
    will be made to form some of the internal coordinates from just stretch
    simple coordinates.  The default is false.

  \item[\keywd{decouple\_bends}] This is like \keywd{decouple\_bonds}
    except it applies to the bend-like coordinates.  The default is false.

  \item[\keywd{max\_update\_disp}] The maximum displacement to be used in
    the displacement to fixed internal coordinates values.  Larger
    displacements will be broken into several smaller displacements and new
    coordinates will be formed for each of these displacments. This is only
    used when \keywd{fixed} and \keywd{have\_fixed\_values} are given.  The
    default is 0.5.

  \item[\keywd{max\_update\_steps}] The maximum number of steps permitted
    to convert internal coordinate displacements to cartesian coordinate
    displacements.  The default is 100.
               
  \item[\keywd{update\_bmat}] Displacements in internal coordinates are
     converted to a cartesian displacements interatively.  If there are
     large changes in the cartesian coordinates during conversion, then
     recompute the $B$ matrix, which is using to do the conversion.  The
     default is false.

  \item[\keywd{only\_totally\_symmetric}] If a simple test reveals that an
     internal coordinate is not totally symmetric, then it will not be
     added to the internal coordinate list.  The default is true.
               
  \item[\keywd{simple\_tolerance}] The internal coordinates are formed as
     linear combinations of simple, redundant internal coordinates.
     Coordinates with coefficients smaller then \keywd{simple\_tolerance}
     will be omitted. The default is 1.0e-3.

  \item[\keywd{cartesian\_tolerance}] The tolerance for conversion of
     internal coordinate displacements to cartesian displacements.  The
     default is 1.0e-12.
               
  \item[\keywd{form:print\_simple}] Print the simple internal coordinates.
     The default is false.
               
  \item[\keywd{form:print\_variable}] Print the variable internal
     coordinates.  The default is false.
               
  \item[\keywd{form:print\_constant}] Print the constant internal
     coordinates.  The default is false.
               
  \item[\keywd{scale\_bonds}] Obsolete.  The default value is 1.0.

  \item[\keywd{scale\_bends}] Obsolete.  The default value is 1.0.
               
  \item[\keywd{scale\_tors}] Obsolete.  The default value is 1.0.

  \item[\keywd{scale\_outs}] Obsolete.  The default value is 1.0.

  \item[\keywd{symmetry\_tolerance}] Obsolete.  The default is 1.0e-5.

  \item[\keywd{coordinate\_tolerance}] Obsolete.  The default is 1.0e-7.

\end{description}


%%%%%%%%%%%%%%%%%%%%%%%%%%%%%%%%%%%%%%%%%%%%%%%%%%%%%%%%%%%%%%%%%%%%%%%%%%

\section{The \clsnm{SymmMolecularCoor} Class}\label{SymmMolecularCoor}\index{SymmMolecularCoor}

The \clsnm{SymmMolecularCoor} class derives from
\clsnmref{IntMolecularCoor}.  It provides a unique set of totally symmetric
internal coordinates.  Giving an \clsnmref{MolecularEnergy} object a
\keywd{coor} is usually the best way to optimize a molecular structure.
However, for some classes of molecules \clsnm{SymmMolecularCoor} doesn't
work very well.  For example, enediyne can cause problems.  In these cases,
cartesian coordinates (obtained by not giving the
\clsnmref{MolecularEnergy} object the \keywd{coor} keyword) might be better
or you can manually specify the coordinates that the
\clsnm{SymmMolecularCoor} object uses with the \keywd{variable} keyword
(see the \clsnmref{IntMolecularCoor} class description).

\begin{description}
  \item[\keywd{change\_coordinates}] If true, the quality of the internal
    coordinates will be checked periodically and if they are beginning to
    become linearly dependent a new set of internal coordinates will be
    computed.  The default is false.

  \item[\keywd{max\_kappa2}] A measure of the quality of the internal
    coordinates.  Values of the 2-norm condition, $\kappa_2$, larger than
    \keywd{max\_kappa2} are considered linearly dependent.  The default is
    10.0.

  \item[\keywd{transform\_hessian}] If true, the hessian will be transformed
    every time the internal coordinates are formed.  The default is true.

\end{description}

%%%%%%%%%%%%%%%%%%%%%%%%%%%%%%%%%%%%%%%%%%%%%%%%%%%%%%%%%%%%%%%%%%%%%%%%%%

\section{The \clsnm{RedundMolecularCoor} Class}\label{RedundMolecularCoor}\index{RedundMolecularCoor}

The \clsnm{RedundMolecularCoor} class derives from
\clsnmref{IntMolecularCoor}.  It provides a redundant set of simple
internal coordinates.

%%%%%%%%%%%%%%%%%%%%%%%%%%%%%%%%%%%%%%%%%%%%%%%%%%%%%%%%%%%%%%%%%%%%%%%%%%

\section{The \clsnm{IntCoorGen} Class}\label{IntCoorGen}\index{IntCoorGen}

The \clsnm{IntCoorGen} class is used to construct a set of internal
coordinates for a molecule.

\begin{description}
  \item[\keywd{molecule}] A \clsnmref{Molecule}
    \htmlref{object}{pkvobject}.  There is no default.

  \item[\keywd{radius\_scale\_factor}] If the distance between two atoms is
      less than the radius scale factor times the sum of the atoms' atomic
      radii, then a bond is placed between the two atoms for the purpose of
      finding internal coordinates.  The default is 1.1.

  \item[\keywd{linear\_bend\_threshold}] A bend angle in degress greater
      than 180 minus this keyword's floating point value is considered a
      linear bend. The default is 5.0.

  \item[\keywd{linear\_tors\_threshold}] The angles formed by atoms a-b-c
      and b-c-d are checked for near linearity.  If an angle in degrees is
      greater than 180 minus this keyword's floating point value, then the
      torsion is classified as a linear torsion. The default is 5.0.

  \item[\keywd{linear\_bend}] Generate \clsnmref{BendSimpleCo} objects
      to describe linear bends.  The default is false.

  \item[\keywd{linear\_lbend}] Generate pairs of \clsnmref{LinIPSimpleCo}
      and \clsnmref{LinIPSimpleCo} objects to describe linear bends.  The
      default is true.

  \item[\keywd{linear\_tors}] Generate \clsnmref{TorsSimpleCo} objects
      to described linear torsions.  The default is false.

  \item[\keywd{linear\_stors}] Generate \clsnmref{ScaledTorsSimpleCo}
      objects to described linear torsions.  The default is true.


  \item[\keywd{extra\_bonds}] This is a \htmlref{vector}{pkvarray} of atom
     numbers, where elements $2 (i-1) + 1$ and $2 i$ specify the atoms
     which are bound in extra bond $i$.  The \keywd{extra\_bonds} keyword
     should only be needed for weakly interacting fragments, otherwise all
     the needed bonds will be found.

\end{description}

%%%%%%%%%%%%%%%%%%%%%%%%%%%%%%%%%%%%%%%%%%%%%%%%%%%%%%%%%%%%%%%%%%%%%%%%%%

\section{The \clsnm{IntCoor} Class}\label{IntCoor}\index{IntCoor}

The \clsnm{IntCoor} abstract class describes an internal coordinate of a
molecule.  The following keywords are recognized in the input:

\begin{description}
  \item[\keywd{label}] A label for the coordinate using only to
     identify the coordinate to the user in printouts.  The default
     is no label.

  \item[\keywd{value}] A value for the coordinate.  In the way that
     coordinates are usually used, the default is to compute a value
     from the cartesian coordinates in a \clsnmref{Molecule} object.

  \item[\keywd{unit}] The unit in which the value is given.  This can be
     \keywd{bohr}, \keywd{anstrom}, \keywd{radian}, and \keywd{degree}.
     The default is \keywd{bohr} for lengths and \keywd{radian} for angles.

\end{description}

%%%%%%%%%%%%%%%%%%%%%%%%%%%%%%%%%%%%%%%%%%%%%%%%%%%%%%%%%%%%%%%%%%%%%%%%%%

\section{The \clsnm{SimpleCo} Class}\label{SimpleCo}\index{SimpleCo}

The \clsnm{SimpleCo} abstract class describes a simple internal coordinate
of a molecule.  The number atoms involved can be 2, 3 or 4 and is
determined by the specialization of \clsnm{SimpleCo}.

There are three ways to specify the atoms involved in the internal
coordinate.  The first way is a shorthand notation, just a vector of a
label followed by the atom numbers (starting at 1) is given.  For example,
a stretch between two atoms, 1 and 2, is given, in the
\clsnmref{ParsedKeyVal} format, as
\begin{alltt}
  stretch<\clsnmref{StreSimpleCo}>: [ R12 1 2 ]
\end{alltt}

The other two ways to specify the atoms are more general.  With them, it is
possible to give parameters for the \clsnm{IntCoor} base class (and thus
give the value of the coordinate).  In the first of these input formats, a
vector associated with the keyword \keywd{atoms} gives the atom numbers.
The following specification for \keywd{stretch} is equivalent to that
above:
\begin{alltt}
  stretch<\clsnmref{StreSimpleCo}>:( label = R12 atoms = [ 1 2 ] )
\end{alltt}

In the second, a vector, \keywd{atom\_labels}, is given along with a
\clsnmref{Molecule} object.  The atom labels are looked up in the
\clsnmref{Molecule} object to find the atom numbers.
The following specification for \keywd{stretch} is equivalent to those
above:
\begin{alltt}
  molecule<\clsnmref{Molecule}>: (
    \{ atom_labels atoms   geometry      \} = \{
          H1         H   [ 1.0 0.0 0.0 ]
          H2         H   [-1.0 0.0 0.0 ] \} )
  stretch<\clsnmref{StreSimpleCo}>:( label = R12
                          atom_labels = [ H1 H2 ]
                          molecule = $molecule )
\end{alltt}

%%%%%%%%%%%%%%%%%%%%%%%%%%%%%%%%%%%%%%%%%%%%%%%%%%%%%%%%%%%%%%%%%%%%%%%%%%

\section{The \clsnm{StreSimpleCo} Class}\label{StreSimpleCo}\index{StreSimpleCo}

The \clsnm{StreSimpleCo} class describes an stretch internal coordinate of a
molecule.  The input is described in the documentation of its parent
class \clsnmref{SimpleCo}.

Designating the two atoms as $a$ and $b$ and their cartesian positions as
$\bar{r}_a$ and $\bar{r}_b$, the value of the coordinate, $r$, is
\[ r = \| \bar{r}_a - \bar{r}_b \| \]

%%%%%%%%%%%%%%%%%%%%%%%%%%%%%%%%%%%%%%%%%%%%%%%%%%%%%%%%%%%%%%%%%%%%%%%%%%

\section{The \clsnm{BendSimpleCo} Class}\label{BendSimpleCo}\index{BendSimpleCo}

The \clsnm{BendSimpleCo} class describes an bend internal coordinate of a
molecule.  The input is described in the documentation of its parent
class \clsnmref{SimpleCo}.

Designating the three atoms as $a$, $b$, and $c$ and their cartesian
positions as $\bar{r}_a$, $\bar{r}_b$, and $\bar{r}_c$, the value of the
coordinate, $\theta$, is given by
\begin{eqnarray*}
 \bar{u}_{ab} &=& \frac{\bar{r}_a - \bar{r}_b}{\| \bar{r}_a - \bar{r}_b \|}\\
 \bar{u}_{cb} &=& \frac{\bar{r}_c - \bar{r}_b}{\| \bar{r}_c - \bar{r}_b \|}\\
 \theta       &=& \arccos ( \bar{u}_{ab} \cdot \bar{u}_{cb} )
\end{eqnarray*}

%%%%%%%%%%%%%%%%%%%%%%%%%%%%%%%%%%%%%%%%%%%%%%%%%%%%%%%%%%%%%%%%%%%%%%%%%%

\section{The \clsnm{TorsSimpleCo} Class}\label{TorsSimpleCo}\index{TorsSimpleCo}

The \clsnm{TorsSimpleCo} class describes an torsion internal coordinate of a
molecule.  The input is described in the documentation of its parent
class \clsnmref{SimpleCo}.

Designating the four atoms as $a$, $b$, $c$, and $d$ and their cartesian
positions as $\bar{r}_a$, $\bar{r}_b$, $\bar{r}_c$, and $\bar{r}_d$, the
value of the coordinate, $\tau$, is given by
\begin{eqnarray*}
 \bar{u}_{ab} &=& \frac{\bar{r}_a - \bar{r}_b}{\| \bar{r}_a - \bar{r}_b \|}\\
 \bar{u}_{cb} &=& \frac{\bar{r}_c - \bar{r}_b}{\| \bar{r}_c - \bar{r}_b \|}\\
 \bar{u}_{cd} &=& \frac{\bar{r}_c - \bar{r}_d}{\| \bar{r}_c - \bar{r}_b \|}\\
 \bar{n}_{abc}&=& \frac{\bar{u}_{ab} \times \bar{u}_{cb}}
                       {\| \bar{u}_{ab} \times \bar{u}_{cb} \|} \\
 \bar{n}_{bcd}&=& \frac{\bar{u}_{cd} \times \bar{u}_{bc}}
                       {\| \bar{u}_{cd} \times \bar{u}_{bc} \|} \\
 s            &=& \cases{1, &if $(\bar{n}_{abc}\times\bar{n}_{bcd})
                                  \cdot \bar{u}_{cb} > 0$;\cr
                         -1, &otherwise.\cr}\\
 \tau       &=& s \arccos ( - \bar{n}_{abc} \cdot \bar{n}_{bcd} )
\end{eqnarray*}

%%%%%%%%%%%%%%%%%%%%%%%%%%%%%%%%%%%%%%%%%%%%%%%%%%%%%%%%%%%%%%%%%%%%%%%%%%

\section{The \clsnm{OutSimpleCo} Class}\label{OutSimpleCo}\index{OutSimpleCo}

The \clsnm{OutSimpleCo} class describes an out-of-plane internal coordinate
of a molecule.  The input is described in the documentation of its parent
class \clsnmref{SimpleCo}.

Designating the four atoms as $a$, $b$, $c$, and $d$ and their cartesian
positions as $\bar{r}_a$, $\bar{r}_b$, $\bar{r}_c$, and $\bar{r}_d$, the
value of the coordinate, $\tau$, is given by
\begin{eqnarray*}
 \bar{u}_{ab} &=& \frac{\bar{r}_a - \bar{r}_b}{\| \bar{r}_a - \bar{r}_b \|}\\
 \bar{u}_{cb} &=& \frac{\bar{r}_b - \bar{r}_c}{\| \bar{r}_c - \bar{r}_b \|}\\
 \bar{u}_{db} &=& \frac{\bar{r}_c - \bar{r}_d}{\| \bar{r}_c - \bar{r}_b \|}\\
 \bar{n}_{bcd}&=& \frac{\bar{u}_{cb} \times \bar{u}_{db}}
                       {\| \bar{u}_{cb} \times \bar{u}_{db} \|} \\
 \phi       &=& \arcsin ( \bar{u}_{ab} \cdot \bar{n}_{bcd} )
\end{eqnarray*}

%%%%%%%%%%%%%%%%%%%%%%%%%%%%%%%%%%%%%%%%%%%%%%%%%%%%%%%%%%%%%%%%%%%%%%%%%%

\section{The \clsnm{LinIPSimpleCo} Class}\label{LinIPSimpleCo}\index{LinIPSimpleCo}

The \clsnm{LinIPSimpleCo} class describes an in-plane component of a linear
bend internal coordinate of a molecule.  The input is described in the
documentation of its parent class \clsnmref{SimpleCo}.  A vector,
$\bar{u}$, given as the keyword \keywd{u}, that is not colinear with either
$\bar{r}_a - \bar{r}_b$ or $\bar{r}_b - \bar{r}_c$ must be provided, where
$\bar{r}_a$, $\bar{r}_b$, and $\bar{r}_c$ are the positions of the first,
second, and third atoms, respectively.

  Usually, \clsnmref{LinIPSimpleCo} is used with a corresponding
\clsnmref{LinOPSimpleCo}, which is given exactly the same \keywd{u}.

Designating the three atoms as $a$, $b$, and $c$ and their cartesian
positions as $\bar{r}_a$, $\bar{r}_b$, and $\bar{r}_c$, the value of the
coordinate, $\theta_i$, is given by
\begin{eqnarray*}
 \bar{u}_{ab} &=& \frac{\bar{r}_a - \bar{r}_b}{\| \bar{r}_a - \bar{r}_b \|}\\
 \bar{u}_{cb} &=& \frac{\bar{r}_b - \bar{r}_c}{\| \bar{r}_c - \bar{r}_b \|}\\
 \theta_i     &=& \pi - \arccos ( \bar{u}_{ab} \cdot \bar{u} )
                      - \arccos ( \bar{u}_{cb} \cdot \bar{u} )
\end{eqnarray*}

%%%%%%%%%%%%%%%%%%%%%%%%%%%%%%%%%%%%%%%%%%%%%%%%%%%%%%%%%%%%%%%%%%%%%%%%%%

\section{The \clsnm{LinOPSimpleCo} Class}\label{LinOPSimpleCo}\index{LinOPSimpleCo}

The \clsnm{LinOPSimpleCo} class describes an out-of-plane component of a
linear bend internal coordinate of a molecule.  The input is described in
the documentation of its parent class \clsnmref{SimpleCo}.  A vector,
$\bar{u}$, given as the keyword \keywd{u}, that is not colinear with either
$\bar{r}_a - \bar{r}_b$ or $\bar{r}_b - \bar{r}_c$ must be provided, where
$\bar{r}_a$, $\bar{r}_b$, and $\bar{r}_c$ are the positions of the first,
second, and third atoms, respectively.

  Usually, \clsnmref{LinOPSimpleCo} is used with a corresponding
\clsnmref{LinIPSimpleCo}, which is given exactly the same \keywd{u}.

Designating the three atoms as $a$, $b$, and $c$ and their cartesian
positions as $\bar{r}_a$, $\bar{r}_b$, and $\bar{r}_c$, the value of the
coordinate, $\theta_o$, is given by
\begin{eqnarray*}
 \bar{u}_{ab} &=& \frac{\bar{r}_a - \bar{r}_b}{\| \bar{r}_a - \bar{r}_b \|}\\
 \bar{u}_{cb} &=& \frac{\bar{r}_b - \bar{r}_c}{\| \bar{r}_c - \bar{r}_b \|}\\
 \bar{n}      &=& \frac{\bar{u} \times \bar{u}_{ab}}
                       {\| \bar{u} \times \bar{u}_{ab} \|}\\
 \theta_o     &=& \pi - \arccos ( \bar{u}_{ab} \cdot \bar{n} )
                      - \arccos ( \bar{u}_{cb} \cdot \bar{n} )
\end{eqnarray*}

%%%%%%%%%%%%%%%%%%%%%%%%%%%%%%%%%%%%%%%%%%%%%%%%%%%%%%%%%%%%%%%%%%%%%%%%%%

\section{The \clsnm{ScaledTorsSimpleCo} Class}\label{ScaledTorsSimpleCo}\index{ScaledTorsSimpleCo}

The \clsnm{ScaledTorsSimpleCo} class describes an scaled torsion internal
coordinate of a molecule.  The scaled torsion is more stable that ordinary
torsions (see the \clsnmref{TorsSimpleCo} class) in describing situations
where one of the torsions plane's is given by three near linear atoms.

Designating the four atoms as $a$, $b$, $c$, and $d$ and their cartesian
positions as $\bar{r}_a$, $\bar{r}_b$, $\bar{r}_c$, and $\bar{r}_d$, the
value of the coordinate, $\tau_s$, is given by
\begin{eqnarray*}
 \bar{u}_{ab} &=& \frac{\bar{r}_a - \bar{r}_b}{\| \bar{r}_a - \bar{r}_b \|}\\
 \bar{u}_{cb} &=& \frac{\bar{r}_c - \bar{r}_b}{\| \bar{r}_c - \bar{r}_b \|}\\
 \bar{u}_{cd} &=& \frac{\bar{r}_c - \bar{r}_d}{\| \bar{r}_c - \bar{r}_b \|}\\
 \bar{n}_{abc}&=& \frac{\bar{u}_{ab} \times \bar{u}_{cb}}
                       {\| \bar{u}_{ab} \times \bar{u}_{cb} \|} \\
 \bar{n}_{bcd}&=& \frac{\bar{u}_{cd} \times \bar{u}_{cb}}
                       {\| \bar{u}_{cd} \times \bar{u}_{cb} \|} \\
 s            &=& \cases{-1, &if $(\bar{n}_{abc}\times\bar{n}_{bcd})
                                  \cdot \bar{u}_{cb} > 0$;\cr
                         1, &otherwise.\cr}\\
 \tau_s       &=& \sqrt{\left(1-(\bar{u}_{ab} \cdot \bar{u}_{cb}\right)^2)
                        \left(1-(\bar{u}_{cb} \cdot \bar{u}_{cd}\right)^2)}
                  \arccos ( - \bar{n}_{abc} \cdot \bar{n}_{bcd} )
\end{eqnarray*}

%%%%%%%%%%%%%%%%%%%%%%%%%%%%%%%%%%%%%%%%%%%%%%%%%%%%%%%%%%%%%%%%%%%%%%%%%%

\section{The \clsnm{SumIntCoor} Class}\label{SumIntCoor}\index{SumIntCoor}

The \clsnm{SumIntCoor} class specifies a linear combination of
\clsnmref{IntCoor} objects.  The following keywords are recognized:

\begin{description}
  \item[\keywd{coor}] A \htmlref{vector}{pkvarray} of \clsnmref{IntCoor}
    \htmlref{objects}{pkvobject} that define the summed coordinates.

  \item[\keywd{coef}] A \htmlref{vector}{pkvarray} of floating point
    numbers that gives the coefficients of the summed coordinates.

\end{description}

The following is a sample \clsnmref{ParsedKeyVal} input for
a \clsnmref{SumIntCoor} object.
\begin{alltt}
  sumintcoor<\clsnmref{SumIntCoor}>: (
    coor: [
      <\clsnmref{StreSimpleCo}>:( atoms = [ 1 2 ] )
      <\clsnmref{StreSimpleCo}>:( atoms = [ 2 3 ] )
      ]
    coef = [ 1.0 1.0 ]
    )
\end{alltt}

%%%%%%%%%%%%%%%%%%%%%%%%%%%%%%%%%%%%%%%%%%%%%%%%%%%%%%%%%%%%%%%%%%%%%%%%%%

\section{The \clsnm{SetIntCoor} Class}\label{SetIntCoor}\index{SetIntCoor}

The \clsnm{SetIntCoor} class describes a set of internal coordinates.  It
can automatically generate these coordinates using a integral coordinate
generator object (see the \clsnmref{IntCoorGen} class) or the internal
coordinates can be explicity given.  Internal coordinate classes include:
\clsnmref{SumIntCoor}, \clsnmref{StreSimpleCo}, \clsnmref{BendSimpleCo},
\clsnmref{LinIPSimpleCo}, \clsnmref{LinOPSimpleCo},
\clsnmref{TorsSimpleCo}, \clsnmref{ScaledTorsSimpleCo}, and
\clsnmref{OutSimpleCo}.

\begin{description}
  \item[\keywd{generator}] A \clsnmref{IntCoorGen}
    \htmlref{object}{pkvobject} that will be used to generate the internal
    coordinates.

  \item[\vrbl{i}] A sequence of integer keywords, all $i$ for $0 \leq i <
    n$, can be assign to \clsnmref{IntCoor} objects.

\end{description}

The following is a sample \clsnmref{ParsedKeyVal} input for
a \clsnmref{SetIntCoor} object.
\begin{alltt}
  setintcoor<\clsnmref{SetIntCoor}>: [
    <\clsnmref{SumIntCoor}>: (
      coor: [
        <\clsnmref{StreSimpleCo}>:( atoms = [ 1 2 ] )
        <\clsnmref{StreSimpleCo}>:( atoms = [ 2 3 ] )
        ]
      coef = [ 1.0 1.0 ]
      )
    <\clsnmref{BendSimpleCo}>:( atoms = [ 1 2 3 ] )
  ]
\end{alltt}

%%%%%%%%%%%%%%%%%%%%%%%%%%%%%%%%%%%%%%%%%%%%%%%%%%%%%%%%%%%%%%%%%%%%%%%%%%

\section{The \clsnm{MolecularFrequencies} Class}\label{MolecularFrequencies}\index{MolecularFrequencies}

The \clsnm{MolecularFrequencies} class is used to generate vibrational
frequencies and compute thermodynamic information for a molecule.

\begin{description}
  \item[\keywd{mole}] A \clsnmref{MolecularEnergy}
    \htmlref{object}{pkvobject}.  If this is not given then
    \keywd{molecule} must be given.

  \item[\keywd{molecule}] A \clsnmref{Molecule}
    \htmlref{object}{pkvobject}.  If this is not given than \keywd{mole}
    must be given.

  \item[\keywd{debug}] An integer which, if nonzero, will cause extra
    output.

  \item[\keywd{displacement}] The amount that coordinates will be
    displaced.  The default is 0.001.

\end{description}

%%%%%%%%%%%%%%%%%%%%%%%%%%%%%%%%%%%%%%%%%%%%%%%%%%%%%%%%%%%%%%%%%%%%%%%%%%

\section{The \clsnm{MolEnergyConvergence} Class}\label{MolEnergyConvergence}\index{MolEnergyConvergence}

The \clsnm{MolEnergyConvergence} class derives from the
\clsnmref{Convergence} class.  The \clsnm{MolEnergyConvergence} class
allows the user to request that cartesian coordinates be used in evaluating
the convergence criteria.  This is useful, since the internal coordinates
can be somewhat arbitary.  If the optimization is constrained, then the
fixed internal coordinates will be projected out of the cartesian
gradients.  The input is similar to that for \clsnmref{Convergence} class
with the exception that giving none of the convergence criteria keywords is
the same as providing the following input:
\begin{alltt}
  conv<\clsnm{MolEnergyConvergence}>: (
    max_disp = 1.0e-4
    max_grad = 1.0e-4
    graddisp = 1.0e-4
  )
\end{alltt}

For \clsnm{MolEnergyConverence} to work, the \clsnmref{Function} object
given to the \clsnmref{Optimizer} object must derive from
\clsnmref{MolecularEnergy}.

The other input parameter is listed below:
\begin{description}
  \item[\keywd{cartesian}] If true, cartesian displacements and gradients
    will be compared to the convergence criteria.  The default is true.

\end{description}
