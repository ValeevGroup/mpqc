
\documentclass{report}

\usepackage{html}
\usepackage{verbatim}
\usepackage{alltt}
\usepackage{makeidx}
\usepackage{scdoc}

\setcounter{tocdepth}{1}

\makeindex

\begin{document}

\title{The SC Programmer's Manual}

\author{Curtis L. Janssen \and Edward T. Seidl \and Ida M. B. Nielsen
        \and Michael E. Colvin}

\date{\today}

\maketitle

\begin{abstract}
The Scientific Computing library (SC) provides C++ class libraries
to assist in scientific computing.

SC can be used on Unix compatible workstations (Intel/Linux, R8000,
RS/6000), symmetric multi-processors (SGI Power Challenge), and
massively parallel computers (IBM SP2, Intel Paragon).
\end{abstract}

\tableofcontents

\part{Utility Libraries}

\input{util/ref/doc/refdoc.tex}

\chapter{The Described Class Library}
\index{metainformation}
\index{cast operations}

The class library provides abstract base classes, macros,
and include files that collectively provide a mechanism that
allows programmers to retrieve information about a class'
name; parents; and void, \clsnmref{StateIn}\srccd{\&}, and
\clsnmref{KeyVal}\srccd{\&} constructors.  Also, a castdown
mechanism is provided.

\subsection{Using \clsnm{DescribedClass}}
\index{CLASSNAME}
\index{PARENTS}
\index{HAVE\_CTOR}
\index{HAVE\_KEYVAL\_CTOR}
\index{HAVE\_STATEIN\_CTOR}

The special nature of described classes requires that the base class,
\clsnmref{DescribedClass}, cannot provide everything needed.  To assist the
user in setting up described classes several include files are provided.
To use these include files, the programmer must define some \exenm{cpp}
macros and, within the body of the class declaration,
include \filnm{util/class/classd.h} for concrete classes and
\filnm{util/class/classda.h} for abstract classes.  Include files
are also provided for implementing all but one of the member functions
needed by each \clsnmref{DescribedClass} descendant.  These are
\filnm{util/class/classi.h} for concrete classes and
\filnm{util/class/classia.h} for abstract classes.

Each of these include files examines several \srccd{cpp}
macros to correctly set up the class declarations and definitions.
All of the macros are automatically
\srccd{undef}'ed by each of the include files.
The macros are:

\begin{description}
\item[\srccd{CLASSNAME}]
  The name of the class.
\item[\srccd{PARENTS}]
  The parents of the class exactly as they appear in the class
  declaration.  For example \srccd{public A, private B, virtual Z},
  without quotes.  The order, access specification, and virtualness
  must be correct or the castdown mechanism could cause erratic behavior.
  This is only needed for the implementation include file.  It should
  not be set for the declaration include file.
\item[\srccd{HAVE\_CTOR}]
  This should be defined if the programmer (or compiler) has provided
  a constructor for the class taking no arguments.
\item[\srccd{HAVE\_KEYVAL\_CTOR}]
  This should be defined if the programmer has provided
  a constructor for the class taking a \clsnmref{KeyVal}\srccd{\&} argument.
  This is useful if objects of this type need to be read from
  an input file.
\item[\srccd{HAVE\_STATEIN\_CTOR}]
  This should be defined if the programmer has provided
  a constructor taking a \clsnm{StateIn}\srccd{\&} arguments.
  This is necessary if objects of this type are to be saved and
  restored.
\end{description}

A class declaration would look like:

\begin{alltt}
class A: virtual public \clsnmref{DescribedClass} \{
#  define CLASSNAME A
#  define HAVE_CTOR
#  include <util/class/classd.h>
\};
\end{alltt}

The source file implementing the members of \clsnm{A} would follow a
similar procedure to define \clsnm{A}'s members.  For example:

\begin{alltt}
#define CLASSNAME A
#define PARENTS virtual public \clsnmref{DescribedClass}
#define HAVE_CTOR
#include <util/class/classi.h>
\end{alltt}

The \srccd{\_castdown} operation must be implemented by the user.
It is very simple to write, but small errors can lead to difficult
to find bugs.  For the example above, the \srccd{\_castdown} operator for
class \srccd{A} would look like:

\begin{alltt}
A::_castdown(const \clsnmref{ClassDesc}*cd)
\{
  void* casts[] =  \{ \clsnmref{DescribedClass}::_castdown(cd) \};
  return do_castdowns(casts,cd);
\}
\end{alltt}

The cast array contains the results of attempting to call
\srccd{\_castdown} for each of the parent classes.  The order and
number of the castings must match those given by the \srccd{PARENTS}
macro.  Most of the work involved in doing a castdown has been
moved into the helper function \srccd{do\_castdowns}, which is
implemented by the provided include files.

\subsection{The \clsnm{DCRef} Macros}
\index{DescribedClass\_REF\_dec}
\index{DescribedClass\_REF\_def}

Some C++ compilers don't handle class templates very well so
\exenm{cpp} macros have been provided to define smart
pointers to \clsnmref{DescribedClass} derivatives.
\srccd{REF\_dec} and \srccd{REF\_def}.  These macros take the
name of the \clsnmref{DescribedClass} class as the only
argument and declare the interface and define the members of
the new \clsnmref{DCRef} class.  The
\srccd{DescribedClass\_REF\_dec} macro takes a class name as
an argument and constructs a new class declaration which is
formed by appending the class name to \srccd{Ref}.  This
macro would usually be invoked in the same header file that
gives the declaration for that reference counted class.  The
implementation of the members for this class are generated
by using the \srccd{DescribedClass\_REF\_def} which also
takes the class name as the argument.  This should be given
in one of the source files relating to that class and should
not be placed in the header file.  The
\srccd{DescribedClass\_REF\_dec} macro must be followed by a
semicolon.

\section{Using \clsnm{DCRef}}
\index{DCRef}

Following is an example of the use of \clsnmref{DCRef}.
Note that inheritance from \clsnmref{VRefCount}
is unnecessary, since \clsnmref{DescribedClass} already is a
\clsnmref{VRefCount} descendant.

\begin{alltt}
#include <util/class/class.h>

class A: virtual public \clsnmref{DescribedClass} \{
 // all the stuff needed for a DescribedClass omitted
\};
DescribedClass_REF_dec(A);

class B: public A \{
 // all the stuff needed for a DescribedClass omitted
\};
DescribedClass_REF_dec(B);

DescribedClass_REF_def(A);
DescribedClass_REF_def(B);

int
main()
\{
  RefA a1(new B);
  RefB b1;

  // a1 and b1 will refer to the same object
  b1 = a1;

  RefA a2(new A)
  RefB b2;

  // b2 will refer to null since a2 does not refer to a B type object.
  b2 = a2;

  return 0;
\}

\end{alltt}


\chapter{The State Library}
\index{persistence}

The state library provides means for objects to save and restore their
state.  Features include:

\begin{itemize}
\item
  Pointers to base types can be saved and restored.
  The exact types of the saved and restored objects will match.
\item
  If the pointer to an object is saved twice, only one copy of the
  object is saved.  When these two pointers are restored they will
  point to the same object.
\item
  If the pointer to an array of a basic type is saved twice,
  only one copy of then
  array is saved.  When these two pointers are restored they will
  point to the same array.  (This feature causes insurmountable
  problems and will be phased out.  Another technique for providing this
  functionality will be provided.)
\item
  Virtual base classes are dealt with in a manner consistent with
  the way C++ treats virtual base classes.
\item
  The library is portable.  Information about object layout for
  particular compiler implementations is not needed.
\end{itemize}

For objects of a class to be savable with this library the class must
inherit \clsnmref{SavableState} which in turn inherits
\clsnmref{DescribedClass}.  Also, a constructor taking a
\srccd{StateIn}\srccd{\&} argument and a
\srccd{save\_data\_state(\clsnmref{StateOut}\&)} member must be provided.  If
the class has virtual base classes, then a
\srccd{save\_vbase\_state(\clsnmref{StateOut}\&)} member must also be
provided.

\subsection{Simple Example}

Here is a simple example of the specification of a client, \clsnm{C},
of \clsnmref{SavableState}:
\begin{alltt}
class C: virtual public \clsnmref{SavableState} \{
# define CLASSNAME C
# define HAVE_STATEIN_CTOR
# include <util/state/stated.h>
# include <util/class/classd.h>
  private:
    int i;
  public:
    C(\clsnmref{StateIn}&);
    void save_data_state(\clsnmref{StateOut}&);
  \};
\end{alltt}

Here is the implementation for the above:
\begin{alltt}
#define CLASSNAME C
#define HAVE_STATEIN_CTOR
#include <util/state/statei.h>
#include <util/class/classi.h>
void* C::_castdown(\clsnmref{ClassDesc}*cd)
  \{
  void* casts[0];
  return do_castdowns(casts);
  \}
void C::save_data_state(\clsnmref{StateOut}&so) \{
  so.put(i);
  \}
C::C(\clsnmref{StateIn}&si): \clsnmref{SavableState}(si,C::class_desc_) \{
  si.get(i);
  \}
\end{alltt}


\subsection{Example with Inheritance}

Here is an example of the specification of \clsnm{C},
where \clsnm{C} nonvirtually inherits from another
\clsnm{SavableState} derivative:
\begin{alltt}
class C: public B \{
# define CLASSNAME C
# define HAVE_STATEIN_CTOR
# include <util/state/stated.h>
# include <util/class/classd.h>
  private:
    int i;
  public:
    C(StateIn&);
    void save_data_state(StateOut&);
  \};
\end{alltt}

Here is the implementation for the above:
\begin{alltt}
#define CLASSNAME C
#define PARENTS public B
#define HAVE_STATEIN_CTOR
#include <util/state/statei.h>
#include <util/class/classi.h>
void* C::_castdown(ClassDesc*cd)
  \{
  void* casts[] = \{ B::_castdown(cd) \};
  return do_castdowns(casts);
  \}
void C::save_data_state(StateOut&so) \{
  B::save_data_state(so);
  so.put(i);
  \}
C::C(StateIn&si): SavableState(si,C::class_desc_), B(si)  \{
  si.get(i);
  \}
\end{alltt}

\subsection{Example with Virtual and Nonvirtual Inheritance}

Here is an example of the specification of \clsnm{C},
where \clsnm{C} nonvirtually inherits from another client of
\clsnm{SavableState} as well as virtually inherits from a client
of \clsnm{SavableState}:
\begin{alltt}
class C: public B,
         virtual public E \{
# define CLASSNAME C
# define HAVE_STATEIN_CTOR
# include <util/state/stated.h>
# include <util/class/classd.h>
  private:
    int i;
  public:
    C(StateIn&);
    void save_data_state(StateOut&);
  \};
\end{alltt}

Here is the implementation for the above:
\begin{alltt}
#define CLASSNAME C
#define PARENTS public B, virtual public E
#define HAVE_STATEIN_CTOR
#include <util/state/statei.h>
#include <util/class/classi.h>
void* C::_castdown(ClassDesc*cd)
  \{
  void* casts[] = \{B::_castdown(cd),E::_castdown(cd)\};
  return do_castdowns(casts);
  \}
void C::save_vbase_state(StateOut&sio) \{
  E::save_data_state(sio);
  \}
void C::save_data_state(StateOut&so) \{
  B::save_parent_state(so);
  so.put(i);
  \}
C::C(StateIn&si): SavableState(si,C::class_desc_), B(si), E(si) \{
  si.get(i);
  \}
\end{alltt}

\subsection{Example with Pointers to SavableStates}

Here is an example where \clsnm{C} has data members which are
pointers to derivatives of \clsnm{SavableState}:
\begin{alltt}
class C: \{
# define CLASSNAME C
# define HAVE_STATEIN_CTOR
# include <util/state/stated.h>
# include <util/class/classd.h>
  private:
    A* ap; // A is also a SavableState
  public:
    C(StateIn&);
    void save_data_state(StateOut&);
  \};
\end{alltt}

Here is the implementation for the above:
\begin{alltt}
#define CLASSNAME C
#define HAVE_STATEIN_CTOR
#include <util/state/statei.h>
#include <util/class/classi.h>
void* C::_castdown(ClassDesc*cd)
  \{
  void* casts[0];
  return do_castdowns(casts);
  \}
void C::save_data_state(StateOut&so) \{
  so.put(ap);
  \}
C::C(StateIn&si): SavableState(si,C::class_desc_) \{
  ap = A::restore_state(si);
  \}
\end{alltt}

\subsection{Example with Pointers to Data}

Here is an example where \clsnm{C} has data members which are
pointers to data:
\begin{alltt}
class C: virtual public SavableState \{
# define CLASSNAME C
# define HAVE_STATEIN_CTOR
# include <util/state/stated.h>
# include <util/class/classd.h>
  private:
    int vecsize;
    double *vec;
    int n1;
    int n2;
    double **array;
  public:
    C(StateIn&);
    void save_data_state(StateOut&);
  \};
\end{alltt}

Here is the implementation for the above:
\begin{alltt}
#define CLASSNAME C
#define HAVE_STATEIN_CTOR
#include <util/state/statei.h>
#include <util/class/classi.h>
void* C::_castdown(ClassDesc*cd)
  \{
  void* casts[0];
  return do_castdowns(casts);
  \}
void C::save_data_state(StateOut&so) \{
  so.put(vecsize);
  so.put(vec,vecsize);

  so.put(n1);
  so.put(n2);
  if (so.putpointer(array)) \{
    for (int i=0; i<n1; i++) \{
      so.put(array[i],n2);
      \}
    \}
  \}
C::C(StateIn&si): SavableState(si,C::class_desc_) \{
  si.get(vecsize);
  si.get(vec);

  si.get(n1);
  si.get(n2);

  int refnum;
  if (refnum = so.getpointer(array)) \{
    array = new double*[n1];
    so.havepointer(refnum,array);
    for (int i=0; i<n1; i++) \{
      si.get(array[i]);
      \}
    \}
  \}
\end{alltt}

\input{util/state/doc/SSRefBase.cls.tex}
\input{util/state/doc/SSRef.cls.tex}

\input{util/state/doc/StateIn.cls.tex}
\input{util/state/doc/StateInFile.cls.tex}
\input{util/state/doc/StateInBin.cls.tex}
\input{util/state/doc/StateInText.cls.tex}
\input{util/state/doc/StateOut.cls.tex}
\input{util/state/doc/StateOutFile.cls.tex}
\input{util/state/doc/StateOutBin.cls.tex}
\input{util/state/doc/StateOutText.cls.tex}
\input{util/state/doc/QCXDR.cls.tex}



\chapter{The Miscellaneous Library}

\input{util/misc/doc/Debugger.cls.tex}


\chapter{The Group Library}

\input{util/group/doc/MessageGrp.cls.tex}
\input{util/group/doc/ProcMessageGrp.cls.tex}
\input{util/group/doc/intMessageGrp.cls.tex}
\input{util/group/doc/ShmMessageGrp.cls.tex}
\input{util/group/doc/MemoryGrp.cls.tex}
\input{util/group/doc/MemoryGrpBuf.cls.tex}
\input{util/group/doc/MsgStateSend.cls.tex}
\input{util/group/doc/MsgStateRecv.cls.tex}
\input{util/group/doc/StateSend.cls.tex}
\input{util/group/doc/StateRecv.cls.tex}
\input{util/group/doc/BcastStateRecv.cls.tex}
\input{util/group/doc/BcastStateSend.cls.tex}
\input{util/group/doc/BcastState.cls.tex}


\chapter{The KeyVal Library}
\index{input}


\section{The \clsnm{ParsedKeyVal} Input Format}
\label{ParsedKeyVal}
\label{KeyVal}
\index{ParsedKeyVal}
\index{KeyVal}

The \clsnm{KeyVal} class provides a means for users to associate keywords
with values.  \clsnm{ParsedKeyVal} is a specialization of \clsnm{KeyVal}
that permits keyword/value associations in text such as an input file or a
command line string.

The package is flexible enough to allow complex structures and arrays as
well as objects to be read from an input file.

\subsection{Assignment}

As an example of the use of \clsnmref{ParsedKeyVal}, consider the following
input:
\begin{verbatim}
x_coordinate = 1.0
y_coordinate = 2.0
x_coordinate = 3.0
\end{verbatim}
Two assignements will be made.  The keyword \verb|x_coordinate| will be
associated with the value \verb|1.0| and the keyword \verb|y_coordinate|
will be assigned to \verb|2.0|.  The third line in the above input
will have no effect since \verb|x_coordinate| was assigned previously.

\subsection{Keyword Grouping}
\label{pkvgroup}

Lets imagine that we have a program which needs to read in the
characteristics of animals.  There are lots of animals so it might be
nice to catagorize them by their family.  Here is a sample format for
such an input file:
\begin{verbatim}
reptile: (
  alligator: (
    legs = 4
    extinct = no
    )
  python: (
    legs = 0
    extinct = no
    )
  )
bird: (
  owl: (
    flys = yes
    extinct = no
    )
  )
\end{verbatim}

This sample illustrates the use of \vrbl{keyword} \verb|=| \vrbl{value}
assignments and the keyword grouping operators \verb|(| and \verb|)|.
The keywords in this example are
\begin{verbatim}
reptile:alligator:legs
reptile:alligator:extinct
reptile:alligator:legs
reptile:python:size
reptile:python:extinct
bird:owl:flys
bird:owl:extinct
\end{verbatim}

The \verb|:|'s occuring in these keywords break the keywords into
smaller logical units called keyword segments.  The sole purpose of this
is to allow persons writing input files to group the input into easy to
read sections.  In the above example there are two main sections, the
reptile section and the bird section.  The reptile section takes the
form \verb|reptile| \verb|:| \verb|(| \vrbl{keyword} \verb|=| \vrbl{value}
assignments \verb|)|.  Each of the keywords found between the
parentheses has the \verb|reptile:| prefix attached to it.  Within each
of these sections further keyword groupings can be used, as many and as
deeply nested as the user wants.

Keyword grouping is also useful when you need many different programs to
read from the same input file.  Each program can be assigned its own
unique section.

\subsection{Array Construction}
\label{pkvarray}

Input for an array is specified in the input by forming a \htmlref{keyword
group}{pkvgroup}.  The name of the group is the name of the array and the
grouped keywords are the integers $i$, such that $0 \leq i < n$, where $n$
is the number of elements in the array.  For example, an array, called
\verb|array|, of length 3 could be given as follows:
\begin{verbatim}
array: (
  0 = 5.4
  1 = 8.9
  2 = 3.7
  )
\end{verbatim}
The numbers \verb|0|, \verb|1|, and \verb|2| in this example are keyword
segments which serve as indices of \verb|array|.  However, this syntax
is somewhat awkward and array construction operators have been provided
to simplify the input for this case.  The following input is equivalent
to the above input:
\begin{verbatim}
array = [ 5.4 8.9 3.7 ]
\end{verbatim}

More complex arrays than this can be imagined.  Suppose an array of
complex numbers is needed.  For example the input
\begin{verbatim}
carray: (
  0: ( r = 1.0  i = 0.0 )
  1: ( r = 0.0  i = 1.0 )
  )
\end{verbatim}
could be written as
\begin{verbatim}
carray: [
  (r = 1.0 i = 0.0)
  (r = 0.0 i = 1.0)
  ]
\end{verbatim}
which looks a bit nicer than the example without array construction
operators.

Furthermore, the array construction operators can be nested in about
every imaginable way.  This allows multidimensional arrays of
complicated data to be represented.  Here is an example of
input for a lower triangular array:
\begin{verbatim}
ltriarray = [ [ 5.4  ]
              [ 0.0 2.8 ]
              [ 0.1 0.0 3.7 ] ]
\end{verbatim}

%It would be nice to just extend an array that is already defined.  This
%feature has not been implemented; however, user demand for it might
%result in its implementation.  The operators reserved for this purpose
%are \verb|+[| and \verb|]|.

\subsection{Table Construction}
\label{pkvtable}

Although the \htmlref{array construction operators}{pkvarray} will suit
most requirements for enumerated lists of data, in some cases the input can
still look ugly.  This can, in some cases, be fixed with the table
construction operators, \verb|{| and \verb|}|.

Suppose a few long vectors of the same length are needed and the data in
the \verb|i|th element of each array is related or somehow belong
together.  If the arrays are so long that the width of a page is
exceeded, then data that should be seen next to each other are no longer
adjacent.  The way this problem can be fixed is to arrange the data
vertically side by side rather than horizontally.  The table
construction operators allows the user to achieve this in a very simple
manner.
\begin{verbatim}
balls: (
  color    = [  red      blue     red   ]
  diameter = [   12       14       11   ]
  material = [  rubber  vinyl   plastic ]
  bounces  = [  yes      no       no    ]
  coordinate = [[ 0.0  0.0  0.0]
                [ 1.0  2.0 -1.0]
                [ 1.0 -1.0  1.0]]
  )
\end{verbatim}
can be written
\begin{verbatim}
balls: (
  { color diameter material bounces     coordinate} =
  {  red     12    rubber    yes     [ 0.0  0.0  0.0]
     blue    14    vinyl     no      [ 1.0  2.0 -1.0]
     red     11    plastic   no      [ 1.0 -1.0  1.0] }
  )
\end{verbatim}
The length and width of the table can be anything the user desires.

\subsection{Value Substitution}
\label{ParsedKeyValvalsub}

Occasionally, a user may need to repeat some value several times in an
input file.  If the value must be changed, it would be nice to only
change the value in one place.  The value substitution feature of
\clsnmref{ParsedKeyVal} always the user to do this.  Any place a value can
occur the user can place a \verb|$|.  Following this a keyword must be
given.  This keyword must have been assigned before the attempt is made
to use its value in a value substitution.

Here is an example illustrating most of the variable substition
features:
\begin{verbatim}
default:linewidth = 130
testsub: (
  ke: (
    ke_1 = 1
    ke_2 = 2
    ke_3: (
      ke_31 = 31
      ke_32 = 32
      )
    )
  kx = $ke
  r1 = 3.0
  r2 = $r1
  linewidth = $:default:linewidth
  )
\end{verbatim}
is the same as specifying
\begin{verbatim}
testsub: (
  ke: (
    ke_1 = 1
    ke_3: (
      ke_31 = 31
      ke_32 = 32
      )
    ke_2 = 2
    )
  linewidth = 130
  r2 = 3.0
  r1 = 3.0
  kx: (
    ke_1 = 1
    ke_2 = 2
    ke_3: (
      ke_31 = 31
      ke_32 = 32
      )
    )
  )
\end{verbatim}
It can be seen from this that value substitution can result in entire
keyword segment hierarchies being copied, as well as simple
substitutions.


\subsection{Expression Evaluation}

Suppose your program requires several parameters \verb|x1|, \verb|x2|,
and \verb|x3|.  Furthermore, suppose that their ratios remain fixed for
all the runs of the program that you desire.  It would be best to
specify some scale factor in the input that would be the only thing that
has to be changed from run to run.  If you don't want to or cannot
modify the program, then this can be done directly with
\clsnmref{ParsedKeyVal} as follows
\begin{verbatim}
scale = 1.234
x1 = ( $:scale *  1.2 )
x2 = ( $:scale *  9.2 )
x3 = ( $:scale * -2.0 )
\end{verbatim}
So we see that to the right of the ``\verb|=|'' the characters
``\verb|(|'' and ``\verb|)|'' are the expression construction operators.
This is in contrast to their function when they are to the left of the
``\verb|=|'', where they are the keyword grouping operators.

The expression must be binary and the data is all converted
to double.  If you use the expression construction operators to produce
data that the program expects to be integer, you will certainly get the
wrong answers (unless the desired value happens to be zero).

\subsection{Objects}
\label{pkvobject}

An instance of an object can be can be specified by surrounding it's
classname with the ``\verb|<|'' and ``\verb|>|'' operators immediately
after the keyword naming the data.

A pointer to a single object can be associated with multiple keywords by
using
\hyperref{value substitution}
         {value substitution (see Section~}
         {)}
         {ParsedKeyValvalsub}.
This is accomplished by holding references to all objects once they are
read in.

Consider a linked list class, \clsnm{A}, which reads from the keyword
\verb|next| a reference to an object of class \clsnm{A}.  Input for such an
object, read from keyword \verb|a1|, follows:
\begin{verbatim}
a1<A>: (
    next<A>: (
        next<B>: (
            bdata = 4
            next<A>:()
            )
        )
    )
a2 = $:a
\end{verbatim}

The \verb|a1| list would contain two \verb|A| objects followed by a
\verb|B| object followed by another \verb|A| object.  The \verb|a2| list
refers to exactly the same object as \verb|a1| (not a copy of
\verb|a1|).




\part{Math Libraries}


\chapter{The Matrix Library}

The scientific computing matrix library (SCMAT) is designed around a set
of matrix abstractions that permit very general matrix implementations.
This flexibility is needed to support diverse computing environments.
For example, this library must support, at a minimum: simple matrices
that provide efficient matrix computations in a uniprocessor
environment, clusters of processors with enough memory to store all
matrices connected by a relatively slow network (workstations on an
LAN), clusters of processors with enough memory to store all matrices
and a fast interconnect network (a massively parallel machine such as
the Intel Paragon), and clusters of machines that don't have enough
memory to hold entire matrices.

The design of SCMAT differs from other object-oriented matrix packages
in two important ways.  First, the matrix classes are abstract base
classes.  No storage layout is defined and virtual function calls must
be used to access individual matrix elements.  This would have a
negative performance impact if users needed to frequently access matrix
elements.  The interface to the matrix classes is hopefully rich enough
to avoid individual matrix element access for any computationally
significant task.  The second major difference is that symmetric
matrices do not inherit from matrices, etc.  The SCMAT user must know
whether a matrix is symmetric at all places it is used if any
performance gain, by virtue of symmetry, is expected.

Dimension information is contained objects of the \clsnmref{SCDimension}
type.  In addition to the simple integer dimension, application specific
blocking information can be provided.  For example, in a quantum chemistry
application, the dimension corresponding to the atomic orbital basis set
will have block sizes that correspond to the shells.  Dimensions are used
to create new matrix or vector objects.

The primary abstract classes are \clsnmref{SCMatrix},
\clsnmref{SymmSCMatrix}, \clsnmref{DiagSCMatrix}, and \clsnmref{SCVector}.
These represent matrices, symmetric matrices, diagonal matrices, and
vectors, respectively.  These abstract classes are specialized into groups
of classes.  For example, the locally stored matrix implementation
specializes the abstract classes to \clsnmref{LocalSCMatrix},
\clsnmref{LocalSymmSCMatrix}, \clsnmref{LocalDiagSCMatrix},
\clsnmref{LocalSCVector}, \clsnmref{LocalSCDimension}, and
\clsnmref{LocalSCMatrixKit}.  These specializations are all designed to
work with each other.  However, a given specialization is incompatible with
other matrix specializations.  An attempt to multiply a local matrix by a
distributed matrix would generate an error at runtime.

Since the different groups of classes do not interoperate, some mechanism
of creating consistent specializations is needed.  This is done with
\clsnmref{SCMatrixKit} objects.  \clsnmref{SCMatrixKit} is an abstract base
type which has specializations that correspond to each group of the matrix
specializations.  It is used to create matrices and vectors from that
group.  For example, the \clsnmref{DistSCMatrixKit} is used to create
objects of type \clsnmref{DistSCMatrix}, \clsnmref{DistSymmSCMatrix},
\clsnmref{DistDiagSCMatrix}, and \clsnmref{DistSCVector}.

The abstract matrix classes and their derivations are usually not directly
used by SCMAT users.  The most convenient classes to use are the smart
pointer classes \clsnmref{RefSCMatrix}, \clsnmref{RefSymmSCMatrix},
\clsnmref{RefDiagSCMatrix}, \clsnmref{RefSCDimension}, and
\clsnmref{RefSCMatrixKit}.  These automatically delete matrix objects when
they are no longer needed.  This is through a reference count mechanism
that is supported by the \clsnmref{VRefCount} base class from which the
abstract matrix classes derive.  The smart pointer classes also have matrix
operations such as \srccd{operator *()}, \srccd{operator -()}, and
\srccd{operator +()} defined as members for convenience.  These forward the
operations to the contained matrix object.  The smart pointer classes also
simplify creation of matrices by providing constructors that take as
arguments one or more \clsnmref{RefSCDimension}'s and a
\clsnmref{RefSCMatrixKit}.  These initialize the smart pointer to contain a
new matrix with a specialization corresponding to that of the
\clsnmref{RefSCMatrixKit}.  Matrix operations not provided by the smart
pointer classes but present as member in the abstract classes can be
accessed with \srccd{operator->()}.

If a needed matrix operation is missing, mechanisms exist to add more
general operations.  Operations which only depend on individual elements of
matrices can be provided by specializations of the \clsnmref{SCElementOp}
class.  Sometimes we need operations on matrices with identical dimensions
that examine each element in one matrix along with the corresponding
element from the other matrix.  This is accomplished with
\clsnmref{SCElementOp2} for two matrices and with \clsnmref{SCElementOp3}
for three.

Other features of SCMAT include run-time type facilities and persistence.
Castdown operations (type conversions from less to more derived objects)
and other run-time type information are provided by the
\clsnmref{DescribedClass} base class.  Persistence is not provided by
inheriting from \clsnmref{SavableState} base clase as is the case with many
other classes in the SC class hierarchies, because it is necessary to save
objects in an implementation independent manner.  If a calculation
checkpoints a matrix on a single processor machine and later is restarted
on a multiprocessor machine the matrix would need to be restored as a
different matrix specialization.  This is handled by saving and restoring
matrices' and vectors' data without reference to the specialization.

The following include files are provided by the matrix library:

\begin{description}
\item[\filnm{matrix.h}]
Usually, this is the only include file needed by users of matrices.  It
declares reference counting pointers to abstract matrices.

If kit for a matrix must be created, or a member specific to an
implementation is needed, then that implementation's header file must be
included.

\item[\filnm{elemop.h}]
This is the next most useful include file.  It defines useful
\clsnmref{SCElementOp}, \clsnmref{SCElementOp2}, and \clsnmref{SCElementOp3}
specializations.

\item[\filnm{abstract.h}]
This include file contains the declarations for abstract classes that
users do not usually need to see.  These include \clsnmref{SCDimension},
\clsnmref{SCMatrix}, \clsnmref{SymmSCMatrix}, \clsnmref{DiagSCMatrix},
\clsnmref{SCMatrixKit}.  This file is currently included by
\filnm{matrix.h}.

\item[\filnm{block.h}]
This file declares \clsnmref{SCMatrixBlock} and specializations.  It
only need be include by users implementing new \clsnmref{SCElementOp}
specializations.

\item[\filnm{blkiter.h}]
This include file declares the implementations of
\clsnmref{SCMatrixBlockIter}.  It only need be include by users implementing
new \clsnmref{SCElementOp} specializations.

\item[\filnm{vector3.h}]
This declares \clsnmref{SCVector3}, a lightweight vector of length three.

\item[\filnm{matrix3.h}]
This declares \clsnmref{SCMatrix3}, a lightweight matrix of dimension three by
three.  It includes \filnm{vector3.h}.

\item[\filnm{local.h}]
This include file is the matrix implementation for locally stored
matrices.  These are suitable for use in a uniprocessor environment.  The
\clsnmref{LocalSCMatrixKit} is the default matrix implementation returned
by the static member \clsnmref{SCMatrixKit}\srccd{::default\_matrixkit}.
This file usually doesn't need to be included.

\item[\filnm{dist.h}]
This include file is the matrix implementation for distributed matrices.
These are suitable for use in a distributed memory multiprocessor which
does not have enough memory to hold all of the matrix elements on each
processor.  This file usually doesn't need to be included.

\item[\filnm{repl.h}]
This include file is the matrix implementation for replicated matrices.
These are suitable for use in a distributed memory multiprocessor which
does have enough memory to hold all of the matrix elements on each
processor.  This file usually doesn't need to be included.

\item[\filnm{blocked.h}]
This include file is the matrix implementation for blocked matrices.
Blocked matrices store a matrix as subblocks that are matrices from another
matrix specialization.  These are used to save storage and computation time
in quantum chemistry applications for molecules with other than $C_1$ point
group symmetry.

\end{description}

\section{Matrix Dimensions}

In addition to the simple integer dimension, objects of the
\clsnmref{SCDimension} class contain application specific blocking
information.  This information is held in an object of class
\clsnmref{SCBlockInfo}.

\input{math/scmat/doc/SCDimension.cls.tex}
\input{math/scmat/doc/SCBlockInfo.cls.tex}

\section{Matrix Reference Classes}\label{matrixrefclass}

The easiest way to use SCMAT is through the smart pointer classes
\clsnmref{RefSCMatrix}, \clsnmref{RefSymmSCMatrix},
\clsnmref{RefDiagSCMatrix}, \clsnmref{RefSCVector},
\clsnmref{RefSCDimension}, and \clsnmref{RefSCMatrixKit}.  These are based
on the \clsnmref{Ref} reference counting package and automatically delete
matrix objects when they are no longer needed.  These reference classes
also have common operations defined as members for convenience.  This makes
it unnecessary to also use the sometimes awkward syntax of
\srccd{operator->()} to manipulate the contained objects.

\input{math/scmat/doc/RefSCDimension.cls.tex}
\input{math/scmat/doc/RefSCVector.cls.tex}
\input{math/scmat/doc/RefSCMatrix.cls.tex}
\input{math/scmat/doc/RefSymmSCMatrix.cls.tex}
\input{math/scmat/doc/RefDiagSCMatrix.cls.tex}

\section{Abstract Matrix Classes}

This section documents the primary abstract classes: \clsnmref{SCMatrix},
\clsnmref{SymmSCMatrix}, \clsnmref{DiagSCMatrix}, and \clsnmref{SCVector},
as well as the \clsnmref{SCMatrixKit} class which allows the programmer to
generate consistent specializations of matrices.  These represent matrices,
symmetric matrices, diagonal matrices, and vectors, respectively.

This section is primarily for implementers of new specializations
of matrices.  Users of existing matrices will be most interested
in \htmlref{the matrix reference classes}{matrixrefclass}.

\input{math/scmat/doc/SCMatrixKit.cls.tex}
\input{math/scmat/doc/SCVector.cls.tex}
\input{math/scmat/doc/SCMatrix.cls.tex}
\input{math/scmat/doc/SymmSCMatrix.cls.tex}
\input{math/scmat/doc/DiagSCMatrix.cls.tex}

\section{Matrix Storage}

All elements of matrices and vectors are kept in blocks.  The
choice of blocks and where they are keep is left up to each
matrix specialization.

\input{math/scmat/doc/SCMatrixBlock.cls.tex}
\input{math/scmat/doc/SCMatrixRectBlock.cls.tex}
\input{math/scmat/doc/SCMatrixRectSubBlock.cls.tex}
\input{math/scmat/doc/SCMatrixLTriBlock.cls.tex}
\input{math/scmat/doc/SCMatrixLTriSubBlock.cls.tex}
\input{math/scmat/doc/SCMatrixDiagBlock.cls.tex}
\input{math/scmat/doc/SCMatrixDiagSubBlock.cls.tex}
\input{math/scmat/doc/SCVectorSimpleBlock.cls.tex}
\input{math/scmat/doc/SCVectorSimpleSubBlock.cls.tex}

\section{Manipulating Matrix Elements with Element Operations}

\input{math/scmat/doc/SCElementOp.cls.tex}
\input{math/scmat/doc/SCElementOp2.cls.tex}
\input{math/scmat/doc/SCElementOp3.cls.tex}
\input{math/scmat/doc/SCMatrixBlockIter.cls.tex}

\section{\clsnm{SCElementOp} Specializations}

Several commonly needed element operations are already coded
up and available by including \filnm{math/scmat/elemop.h}.
Below are descriptions of these classes:

\begin{description}
\item[\clsnm{SCElementScalarProduct}] This \clsnmref{SCElementOp2} computes
the scalar product of two matrices or vectors.  The result is available
after the operation from the return value of the \srccd{result()} member.
\item[\clsnm{SCDestructiveElementProduct}] This \clsnmref{SCElementOp2}
replaces the elements of the matrix or vector whose \srccd{element\_op}
member is called.  The resulting values are the element by element products
of the two matrices or vectors.
\item[\clsnm{SCElementScale}] This scales each element by an amount given
in the constructor.
\item[\clsnm{SCElementRandomize}] This generates random elements.
\item[\clsnm{SCElementAssign}] Assign to each element the value passed to
the constructor.
\item[\clsnm{SCElementSquareRoot}] Replace each element with its square
root.
\item[\clsnm{SCElementInvert}] Replace each element by its reciprocal.
\item[\clsnm{SCElementScaleDiagonal}] Scales the diagonal elements of a
matrix by the argument passed to the constructor.  Use of this on a vector
is undefined.
\item[\clsnm{SCElementShiftDiagonal}] Add the value passed to the
constructor to the diagonal elements of the matrix.  Use of this on a
vector is undefined.
\item[\clsnm{SCElementMaxAbs}] Find the maximum absolute value element in a
matrix or vector.  The result is available as the return value of the
\srccd{result()} member.
\item[\clsnm{SCElementDot}] The constructor for this class takes three
arguments: \linebreak \srccd{SCElementDot(double**\vrbl{a},
double**\vrbl{b}, int \vrbl{length})}.  The length of each vector given by
\vrbl{a} and \vrbl{b} is given by \vrbl{length}.  The number of vectors in
\vrbl{a} is the number of rows in the matrix and the number in \vrbl{b} is
the number of columns.  To each element in the matrix $m_{ij}$ the dot
product of the $a_i$ and $b_j$ is added.
\item[\clsnm{SCElementAccumulateSCMatrix}]  This is obsolete---do not use it.
\item[\clsnm{SCElementAccumulateSymmSCMatrix}] This is obsolete---do not
use it.
\item[\clsnm{SCElementAccumulateDiagSCMatrix}] This is obsolete---do not
use it.
\item[\clsnm{SCElementAccumulateSCVector}] This is obsolete---do not use
it.
\end{description}

\section{Manipulating Matrix Elements with Block Iterators}

\input{math/scmat/doc/SCMatrixSubblockIter.cls.tex}

\subsection{Local Matrices}

Local matrices do no communication.  All elements reside on each node
and all computations are duplicated on each node.

\subsection{Replicated Matrices}

Replicated matrices hold all of the elements on each node, however
do some communications in order to reduce computation time.

\subsection{Distributed Matrices}

Distributed matrices spread the elements across all the nodes and
thus require less storage than local matrices however these use
more communications than replicated matrices.

\subsection{Blocked Matrices}

Blocked matrices are used to implement point group symmetry.  Another
matrix specialization is used to hold the diagonal subblocks of a
matrix.  The offdiagonal subblocks are known to be zero and not stored.
This results in considerable savings in storage and computation for
those cases where it applies.


\chapter{The Optimize Library}
\index{optimization}

\input{math/optimize/doc/Function.cls.tex}
\input{math/optimize/doc/Optimize.cls.tex}
\input{math/optimize/doc/Convergence.cls.tex}
\input{math/optimize/doc/NonlinearTransform.cls.tex}
\input{math/optimize/doc/IdentityTransform.cls.tex}


\chapter{The Symmetry Library}
\index{symmetry}

The symmetry library implements a few basic ideas in group theory.  Namely
it provides a character table for a given point group.  Currently it does
not support the cubic point groups (e.g. $T_d$ or $I_h$).  The only classes
intended for public consumption are the \clsnmref{CharacterTable} and
\clsnmref{PointGroup} classes.  \clsnmref{PointGroup} stores the
Schoenflies symbol for a point group, and can generate a
\clsnmref{CharacterTable} on the fly.  A \clsnmref{CharacterTable} contains
\clsnmref{IrreducibleRepresentation}'s and \clsnmref{SymmetryOperation}'s.

The following include files are provided by the symmetry library:

\begin{description}
\item[\filnm{pointgroup.h}]
This is the only include file for the symmetry library.  It provides
definitions for all of the symmetry-related classes.
\end{description}

\input{math/symmetry/doc/SymmetryOperation.cls.tex}
\input{math/symmetry/doc/SymRep.cls.tex}
\input{math/symmetry/doc/IrreducibleRepresentation.cls.tex}
\input{math/symmetry/doc/CharacterTable.cls.tex}
\input{math/symmetry/doc/PointGroup.cls.tex}


\part{Chemistry Libraries}


\chapter{The Molecule Library}
\index{molecules}

\input{chemistry/molecule/doc/ChemicalElement.cls.tex}
\input{chemistry/molecule/doc/Molecule.cls.tex}
\input{chemistry/molecule/doc/SimpleCo.cls.tex}
\input{chemistry/molecule/doc/StreSimpleCo.cls.tex}
\input{chemistry/molecule/doc/BendSimpleCo.cls.tex}
\input{chemistry/molecule/doc/TorsSimpleCo.cls.tex}
\input{chemistry/molecule/doc/ScaledTorsSimpleCo.cls.tex}
\input{chemistry/molecule/doc/OutSimpleCo.cls.tex}
\input{chemistry/molecule/doc/LinIPSimpleCo.cls.tex}
\input{chemistry/molecule/doc/LinOPSimpleCo.cls.tex}
\input{chemistry/molecule/doc/IntCoor.cls.tex}
\input{chemistry/molecule/doc/SumIntCoor.cls.tex}
\input{chemistry/molecule/doc/SetIntCoor.cls.tex}
\input{chemistry/molecule/doc/IntCoorGen.cls.tex}
\input{chemistry/molecule/doc/MolecularCoor.cls.tex}
\input{chemistry/molecule/doc/IntMolecularCoor.cls.tex}
\input{chemistry/molecule/doc/SymmMolecularCoor.cls.tex}
\input{chemistry/molecule/doc/RedundMolecularCoor.cls.tex}
\input{chemistry/molecule/doc/CartMolecularCoor.cls.tex}
\input{chemistry/molecule/doc/AtomicCenter.cls.tex}
\input{chemistry/molecule/doc/MolecularFormula.cls.tex}


\printindex

\end{document}
