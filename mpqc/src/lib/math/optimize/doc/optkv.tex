
%%%%%%%%%%%%%%%%%%%%%%%%%%%%%%%%%%%%%%%%%%%%%%%%%%%%%%%%%%%%%%%%%%%%%%%%%%

\section{The \clsnm{Function} Class}\label{Function}\index{Function}

The \clsnm{Function} abstract class computes the value and
derivatives of a function for a given set of input
parameters.  Its \clsnmref{KeyVal} constructor reads the
following information:

\begin{description}
  \item[\keywd{matrixkit}] Gives a \clsnmref{SCMatrixKit} object.
    If it is not specified, a default \clsnmref{SCMatrixKit}
    is selected.

  \item[\keywd{value\_accuracy}] Sets the accuracy to which values are
    computed.  The default is the machine accuracy.

  \item[\keywd{gradient\_accuracy}] Sets the accuracy to which gradients
    are computed.  The default is the machine accuracy.

  \item[\keywd{hessian\_accuracy}] Sets the accuracy to which hessians are
    computed.  The default is the machine accuracy.

\end{description}

%%%%%%%%%%%%%%%%%%%%%%%%%%%%%%%%%%%%%%%%%%%%%%%%%%%%%%%%%%%%%%%%%%%%%%%%%%

\section{The \clsnm{Optimize} Class}\label{Optimize}\index{Optimize}

The \clsnm{Optimize} abstract class find stationary points for the values
of \clsnmref{Function} objects.  Its \clsnmref{KeyVal} constructor reads
the following information:

\begin{description}
  \item[\keywd{checkpoint}] If true, the optimization will be checkpointed.
     The default is false.

  \item[\keywd{checkpoint\_file}] The name of the checkpoint file.
     The name defaults to \filnm{opt\_ckpt.dat}.

  \item[\keywd{max\_iterations}] The maximum number of interations.
     The default is 10.

  \item[\keywd{max\_stepsize}] The maximum stepsize.  The default is 0.6.

  \item[\keywd{function}] A \clsnmref{Function} object.  There is
     no default.

  \item[\keywd{convergence}] This can be either a floating point number
     or a \clsnmref{Convergence} object.  If it is a floating point
     number then it is the convergence criterion.  See the description
     \clsnmref{Convergence} class for the default.

\end{description}

%%%%%%%%%%%%%%%%%%%%%%%%%%%%%%%%%%%%%%%%%%%%%%%%%%%%%%%%%%%%%%%%%%%%%%%%%%

\section{The \clsnm{Convergence} Class}\label{Convergence}\index{Convergence}

The \clsnm{Convergence} class is used by the optimizer to determine
when an optimization is converged.  The \clsnmref{KeyVal} input for
\clsnm{Convergence} is given below.  Giving none of these keywords 
is the same as giving the following input:
\begin{alltt}
  conv<\clsnm{Convergence}>: (
    max_disp = 1.0e-6
    max_grad = 1.0e-6
    graddisp = 1.0e-6
  )
\end{alltt}

\begin{description}
  \item[\keywd{max\_disp}] The value of the maximum displacement must be
     less then the value of this keyword for the calculation to be
     converged.  The default is to not check this parameter.  However, if
     no other keyword are given, default convergence parameters are chosen
     as described above.

  \item[\keywd{max\_grad}] The value of the maximum gradient must be less
     then the value of this keyword for the calculation to be converged.
     The default is to not check this parameter.  However, if no other
     keyword are given, default convergence parameters are chosen as
     described above.

  \item[\keywd{rms\_disp}] The value of the RMS of the displacements must
     be less then the value of this keyword for the calculation to be
     converged.  The default is to not check this parameter.  However, if
     no other keyword are given, default convergence parameters are chosen
     as described above.

  \item[\keywd{rms\_grad}] The value of the RMS of the gradients must be
     less then the value of this keyword for the calculation to be
     converged.  The default is to not check this parameter.  However, if
     no other keyword are given, default convergence parameters are chosen
     as described above.

  \item[\keywd{graddisp}] The value of the scalar product of the gradient
     vector with the displacement vector must be less then the value of
     this keyword for the calculation to be converged.  The default is to
     not check this parameter.  However, if no other keyword are given,
     default convergence parameters are chosen as described above.

\end{description}

%%%%%%%%%%%%%%%%%%%%%%%%%%%%%%%%%%%%%%%%%%%%%%%%%%%%%%%%%%%%%%%%%%%%%%%%%%

\section{The \clsnm{QNewtonOpt} Class}\label{QNewtonOpt}\index{QNewtonOpt}

The \clsnm{QNewtonOpt} class derives from \clsnmref{Optimize}.  It
implements a quasi-Newton optimization scheme.

\begin{description}
  \item[\keywd{update}] This gives a \clsnmref{HessianUpdate} object.
     The default is to not update the hessian.

  \item[\keywd{hessian}] By default, the guess hessian is obtained from the
     \clsnmref{Function} object.  This keyword specifies an lower triangle
     array (the second index must be less than or equal to than the first)
     that replaces the guess hessian.  If some of the elements are not
     given, elements from the guess hessian will be used.

  \item[\keywd{lineopt}] This gives a \clsnmref{LineOpt} object for doing
     line optimizations in the Newton direction.  The default is to skip
     the line optimizations.

  \item[\keywd{accuracy}] The accuracy with which the first gradient will
     be computed.  If this is two large, it may be necessary to evaluate
     the first gradient point twice.  If it is two small, it may take
     longer to evaluate the first point. The default is 0.0001.

  \item[\keywd{print\_x}] If true, print the coordinates each iteration.
     The default is false.

  \item[\keywd{print\_gradient}] If true, print the gradient each
    iteration. The default is false.

  \item[\keywd{print\_hessian}] If true, print the approximate hessian each
    iteration. The default is false.

\end{description}

%%%%%%%%%%%%%%%%%%%%%%%%%%%%%%%%%%%%%%%%%%%%%%%%%%%%%%%%%%%%%%%%%%%%%%%%%%

\section{The \clsnm{EFCOpt} Class}\label{EFCOpt}\index{EFCOpt}

The \clsnm{EFCOpt} class derives from \clsnmref{Optimize}.  It implements
eigenvector following as described by Baker in J. Comput. Chem., Vol 7, No
4, 385-395, 1986.

\begin{description}
  \item[\keywd{update}] This gives an \clsnmref{HessianUpdate} object.
     The default is to not update the hessian.

  \item[\keywd{transition\_state}] If this is true than a transition
     state search will be performed. The default is false.

  \item[\keywd{mode\_following}] The default is false.

  \item[\keywd{hessian}] By default, the guess hessian is obtained from the
     \clsnmref{Function} object.  This keyword specifies an lower triangle
     array (the second index must be less than or equal to than the first)
     that replaces the guess hessian.  If some of the elements are not
     given, elements from the guess hessian will be used.

  \item[\keywd{accuracy}] The accuracy with which the first gradient will
     be computed.  If this is two large, it may be necessary to evaluate
     the first gradient point twice.  If it is two small, it may take
     longer to evaluate the first point. The default is 0.0001.

\end{description}

%%%%%%%%%%%%%%%%%%%%%%%%%%%%%%%%%%%%%%%%%%%%%%%%%%%%%%%%%%%%%%%%%%%%%%%%%%

\section{The \clsnm{HessianUpdate} Class}\label{HessianUpdate}\index{HessianUpdate}

The \clsnm{HessianUpdate} abstract class is used to specify a
hessian update scheme.  It is used, for example, by
\clsnmref{QNewtonOpt} objects.

%%%%%%%%%%%%%%%%%%%%%%%%%%%%%%%%%%%%%%%%%%%%%%%%%%%%%%%%%%%%%%%%%%%%%%%%%%

\section{The \clsnm{DFPUpdate} Class}\label{DFPUpdate}\index{DFPUpdate}

The \clsnm{DFPUpdate} class is derived from \clsnmref{HessianUpdate} and
used to specify a Davidson, Fletcher, and Powell hessian update scheme.

\begin{description}
  \item[\keywd{xprev}] The previous coordinates can be given (but is not
    recommended).  The default is none.

  \item[\keywd{gprev}] The previous gradient can be given (but is not
    recommended).  The default is none.

\end{description}

%%%%%%%%%%%%%%%%%%%%%%%%%%%%%%%%%%%%%%%%%%%%%%%%%%%%%%%%%%%%%%%%%%%%%%%%%%

\section{The \clsnm{BFGSUpdate} Class}\label{BFGSUpdate}\index{BFGSUpdate}

The \clsnm{DFPUpdate} class is derived from \clsnmref{DFPUpdate} and used
to specify a Broyden, Fletcher, Goldfarb, and Shanno hessian update scheme.
This hessian update method is the recommended method for use with
\clsnmref{QNewtonOpt} objects.

%%%%%%%%%%%%%%%%%%%%%%%%%%%%%%%%%%%%%%%%%%%%%%%%%%%%%%%%%%%%%%%%%%%%%%%%%%

\section{The \clsnm{PowellUpdate} Class}\label{PowellUpdate}\index{PowellUpdate}

The \clsnm{PowellUpdate} class is derived from \clsnmref{HessianUpdate}
used to specify a Powell hessian update.  This hessian update method is the
recommended method for use with transition state searches (the
\clsnmref{EFCOpt} class can be used for transition state searches).

%%%%%%%%%%%%%%%%%%%%%%%%%%%%%%%%%%%%%%%%%%%%%%%%%%%%%%%%%%%%%%%%%%%%%%%%%%

\section{The \clsnm{LineOpt} Class}\label{LineOpt}\index{LineOpt}

The \clsnm{LineOpt} abstract class derives from \clsnmref{Optimize} and is
used to perform one dimensional optimizations.  However, there are
currently no implementations.

%%%%%%%%%%%%%%%%%%%%%%%%%%%%%%%%%%%%%%%%%%%%%%%%%%%%%%%%%%%%%%%%%%%%%%%%%%

\section{The \clsnm{SelfConsistentExtrapolation} Class}
\label{SelfConsistentExtrapolation}\index{SelfConsistentExtrapolation}

The \clsnm{SelfConsistentExtrapolation} abstract class is used
to iteratively solve equations requiring a self consistent solution,
such as,
\[ \bar{x}' = f(\bar{x}) \]
The only input parameter is \keywd{tolerance}, which is usually not needed
since the objects using \clsnm{SelfConsistentExtrapolation} should set the
tolerances as needed.
%%%%%%%%%%%%%%%%%%%%%%%%%%%%%%%%%%%%%%%%%%%%%%%%%%%%%%%%%%%%%%%%%%%%%%%%%%

\section{The \clsnm{DIIS} Class}
\label{DIIS}\index{DIIS}

The \clsnm{DIIS} class derives from \clsnmref{SelfConsistentExtrapolation}
and provides DIIS extrapolation.  The following keywords are recognized.

\begin{description}
  \item[\keywd{n}] This integer maximum number of data sets to retain.
     The default is 5.

  \item[\keywd{start}] The DIIS extrapolation will begin on the iteration
     given by this integer.  The default is 1.

  \item[\keywd{damping\_factor}] This nonnegative floating point number is
     used to dampen the DIIS extrapolation.  The default is 0.0.

\end{description}
